\setlength{\parskip}{1em} %Sets space between paragraphs
\setlength{\parindent}{0pt} %prevent paragraph indentation
\section{Introducción}
Desde los inicios de la medicina, la historia clínica del paciente es el instrumento de recolección de información primaria para el establecimiento de diagnósticos precisos y formular tratamientos adecuados para el paciente. En la propedéutica clínica, este recurso ha evolucionado de manera sistemática para documentar los antecedentes, síntomas y evolución de las enfermedades con el objetivo de crear un archivo individual que permite la continuidad de la atención al paciente. Así como su uso para la enseñanza médica y la investigación científica. 

En la actualidad, la recopilación de datos provenientes de múltiples fuentes para la integración del historial clínico de un paciente, es una labor que en consulta externa (es decir, no hospitalaria) tiene que realizar el paciente de manera independiente y no se cuenta con un estándar para la organización de un expediente que puede constar de informes de consultas previas, resultados de laboratorio, imagenología, etc. Asimismo, al ser atendido por diversos especialistas para casos de padecimientos que lo requieren, es muy común que en la práctica, se haga este ejercicio en cada consulta donde el médico tratante recopila de nuevo la información, lo cual crea duplicidad y pérdida de tiempo que puede ser empleado en el análisis de los hallazgos anteriores para mejorar la calidad del diagnóstico que se va a proporcionar. 

Lo anterior, pone en evidencia las limitaciones en los modelos tradicionales del manejo de la historia clínica. La fragmentación, la falta de interoperabilidad y colaboración entre los médicos y la repetición del interrogatorio inicial para recoger la misma información que ya se realizó con otros especialistas, en vez de utilizar ese tiempo para que el médico realice una valoración en lugar de una exploración de nuevo, han motivado la búsqueda de soluciones tecnológicas que promuevan una visión más integral, eficiente y segura del proceso de la propedéutica clínica.

En este contexto, el presente anteproyecto de investigación, propone el estudio de un \textit{Modelo Colaborativo Asistido por Computadora para la Integración y Disposición del Historial Clínico de los Pacientes} cuyo objetivo sea mejorar la toma de decisiones médicas, optimizar el trabajo en equipo entre profesionales de la salud y reducir los errores diagnósticos que se pueden generar por la fragmentación de la información, mediante el uso de herramientas digitales, inteligencia artificial y entornos colaborativos. Se pretende estudiar un modelo que facilite el acceso oportuno a la información clínica relevante y promueva la participación activa de todos los actores involucrados en la salud del paciente. Asimismo, se hará un análisis multidisciplinar donde se explorarán aspectos importantes del modelo en temas de seguridad informática, interfaces de usuario adecuadas para cada tipo de usuario y los aspectos legales de resguardo de datos personales.
    \subsection{Antecedentes}
    La evolución del historial clínico refleja la relación dinámica entre la práctica de la medicina, los avances tecnológicos y las necesidades sociales. Desde registros manuscritos a sistemas sofisticados de historial clínico electrónico. Este avance ha marcado hitos en la historia de la ciencia médica como cambios en las prácticas documentales, disposiciones jurídicas y éticas y el impacto de la digitalización de la atención en la salud.

    La idea de obtener información del paciente previo a su diagnóstico y tratamiento, se remonta a la antigua Grecia (siglo V A.C.) En la escuela médica de Hipócrates, considerado el padre de la medicina. Para poder entender mejor esta parte de la propedéutica clínica, es necesario entender dos partes fundamentales como lo explica \cite{llanio_navarro_propedeutica_2007}, que son: \textit{Semiología} y \textit{Sindromología}.
  
    \textit{Semiología}, dado su origen griego es una palabra compuesta (\textit{semeyon}: signo; \textit{logos}: discurso), que significa el estudio de los signos. Es decir, es la parte de la propedéutica que estudia todos aquellos fenómenos que por su propia naturaleza o por simple convención evocan la idea de enfermedad. En palabras más simples, la \textit{Semiología} se encarga de identificar, describir clasificar e interpretar los signos (Manifestación objetiva que el médico puede observar) y los síntomas (Expresión subjetiva que el paciente experimenta y relata)
    
    \textit{Sindromología}, que es el estudio de los síndromes. Esto es, un grupo de síntomas y signos que se presentan formando un cuadro cínico que le da individualidad pero que puede originarse en distintas causas. Por ejemplo en síndrome ictérico que indica alguna enfermedad relacionada con el hígado.

   Como antecedente, los métodos para registrar la historia clínica del paciente van desde las anotaciones narrativas utilizadas para documentar síntomas, tratamientos y resultados con fines principalmente científicos y académicos que realizaba la escuela Hipocrática. En la edad media, se hacían diversos reportes muy rudimentarios con fines más administrativos. Al inicio del siglo XIX, los hospitales empezaron a realizar una estandarización de la información clínica del paciente en "archivos de caso" y registros que generaron una mejora en la atención al enfermo y mayor claridad para los casos de estudio.

    El siglo XX fue testigo de un cambio a una documentación más estructurada y rigurosa de los registros del paciente. Como señala \cite{jaroudi_remembering_2019} en 1964, el Doctor Lawrence Weed, introdujo el Registro Médico Orientado al Problema, que organiza la información en un formato sistemático: Subjetivo, Objetivo, Análisis y Plan (SOAP). Con esto se mejoró la claridad y consistencia en la documentación clínica que funda la base para los futuros sistemas digitales.

    El nacimiento de la computación moderna en los 70's y 80's revolucionó la forma en la que se puede registrar la información y es la era temprana de los registros médicos electrónicos, principalmente fueron adaptados para usos académicos y en hospitales grandes. Estos sistemas hicieron más eficiente el almacenamiento, búsqueda y la forma de compartir datos importantes del paciente. en los 90's esto se expandió a la colaboración de diversos especialistas para socializar en tiempo real la información del paciente. Con esto, mejora la coordinación del personal médico para el tratamiento y dar soporte a la toma de decisiones clínicas.

    La digitalización de los registros de los pacientes introdujo retos en el ámbito legal y ético. Las principales preocupaciones son el resguardo de datos personales, la privacidad de los registros, la seguridad del resguardo y el consentimiento informado.

    El impacto de la digitalización de estos registros ha transformado la forma de brindar atención por parte del personal de salud. Estos han permitido transmitir información entre los especialistas para su análisis y mejorar la calidad de los diagnósticos basados en una fuente de información consistente, así como su interacción con los pacientes. Sin embargo, existen áreas de oportunidad para la mejora de estos sistemas como la interoperabilidad entre ellos, reducir los riesgos de seguridad para la fuga de información, la actualización continua y la portabilidad de estos registros de centros hospitalarios a la consulta externa o preventiva.
    \subsection{Definición del problema}
    Como se describe en la introducción y se detalla en los antecedentes, los modelos colaborativos para la integración y disposición del historial clínico de los pacientes, son una realidad en los ámbitos hospitalarios. Sin embargo, la mayoría de la población lleva consultas médicas externas con especialistas para tratar enfermedades que no necesariamente requieren de una hospitalización. Es en esta parte de la medicina externa donde se encuentra la oportunidad de estudiar los beneficios de un modelo como este. Donde se colecte de manera integral información del paciente con respecto a sus antecedentes y que esta pueda consultarse con autorización de la persona al médico tratante para conocer sus antecedentes como resultados de laboratorio, imagenología, informes de otros especialistas, su \emph{anamnesis}, entre otros.

    En materia de Ciencias Computacionales, lo importante es entender por medio de la investigación, cuales serían las interfaces que mejor se adapten a la visualización y alimentación de dicho modelo. Conocer más a fondo la teoría propedéutica de la medicina en consulta, así como los procesos de análisis de la información por parte de los médicos, es relevante para el diseño de patrones que permitan la interoperabilidad con sistemas existentes, su integración y disposición. Así como la certificación de los datos agregados por los especialistas y la seguridad del resguardo de los registros y el control de su acceso.

    Con lo anterior, las pregunta para el planteamiento del problema de la investigación serían las siguientes:

    ¿En que medida la implementación de un Modelo Colaborativo Asistido por computadora para la Integración y disposición del historial clínico de los pacientes creará beneficios a los pacientes para la comunicación con los diversos especialistas de la salud que consultan?

    ¿Cual es la probabilidad de que este modelo sea una herramienta que funcione en beneficio del personal médico para tener un mejor sustento documental para la toma de decisiones clínicas en la consulta externa y su colaboración con otros especialistas?

    En el marco del diseño. ¿Como se relacionarán los diversos diseños de interfaces para los distintos usuarios del modelo con la satisfacción del uso del modelo y su interpretación?
    \subsection{Justificación}
    El presente anteproyecto justifica la investigación con base en la modernización y la aplicación de la tecnología para mejorar la calidad de vida de las personas. Es cierto que con los avances en la medicina, las personas gozan de buena salud. Sin embargo, dependiendo de la demografía, edad, situación económica, etc. La gran mayoría de las personas tienen que atender algún padecimiento o enfermedad por medio de un médico particular o de atención primaria (fuera del sistema de salud público).
    
    La investigación de este modelo de forma interdisciplinaria, pretende establecer las bases para el diseño de interfaces, arquitectura, patrones y sistemas de seguridad y acceso para una plataforma que pueda ser de utilidad para las personas que requieren servicios de salud por diferentes actores, tengan la facilidad de poder disponer de su información de manera segura y controlada para poder compartirla con el médico tratante y éste pueda analizar de forma colaborativa con otros especialistas, las pruebas de laboratorio, imágenes, diagnósticos previos, antecedentes demográficos, sociales, económicos del paciente y poder certificar la aportación a dicho reporte con actualizaciones realizadas de manera segura y en tiempo real, para garantizar la integridad de la información y con ello obtener una herramienta que facilitará la interacción entré el doctor y el paciente y la toma de decisiones.
    \subsection{Objetivo General}    
    Describir un modelo computacional que cumpla con los criterios de calidad, seguridad, funcionalidad, interoperabilidad, usabilidad, accesibilidad y escalabilidad, para la integración del historial clínico de pacientes en consulta externa, de manera estructurada y estandarizada, para mejorar la administración de la salud para el paciente y mejorar la calidad de los diagnósticos y estudios realizados por el médico.
    \subsection{Objetivos Específicos}
    \begin{itemize}
        \item Demostrar que el úso de éste modelo en la consulta externa mejora la capacidad de análisis y la certeza de los diagnósticos elaborados por los médicos.
        
        \item Comprobar que este modelo es funcional para los pacientes al permitirles tener un acceso seguro y controlado a toda la información de su historial médico.
        
        \item Comparar los diversos sistemas y modelos que utilizan actualmente los médicos para obtener, capturar, almacenar y disponer del expediente clínico de sus pacientes.
        
        \item Describir, con base en la investigación realizada, los patrones de diseño e interfaces que sean óptimos para el análisis de la información del paciente.
        
        \item Definir los criterios de calidad y seguridad para la protección de datos personales y evitar vicios en el uso del modelo.
        
        \item Evaluar la adaptabilidad del modelo a los procesos que actualmente se llevan a cabo en la consulta externa.
    \end{itemize}
    \subsection{Hipótesis}
    El uso del modelo colaborativo asistido por computadora para la integración y disposición del historial clínico de los pacientes mejorará la organización documental y la comunicación del paciente con el médico y a su vez, creará una plataforma de información para el análisis y estudio del caso por parte del especialista para mejorar la calidad de su diagnóstico.