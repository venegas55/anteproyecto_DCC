\section{Metodología Científica}
En este capítulo se abordarán los aspectos metodológicos de la investigación, incluyendo el diseño del estudio, las fases de la investigación, los participantes involucrados, los instrumentos utilizados y el procedimiento seguido.

En un principio, la investigación con relación al tema de estudio se realiza con un enfoque cuantitativo, donde se evalúa en la investigación la variabilidad en cuanto a la aceptación del uso del modelo descrito y sus resultados como una herramienta para la mejora de la calidad en la interacción paciente-medico, así como los beneficios que este genere para la interpretación de la semiología del paciente y mejorar la calidad del diagnóstico médico.

Por lo anterior, la investigación tiene un carácter de alcance \textit{descriptivo y correlacional}, ya que se busca describir las características del modelo propuesto y su relación con la mejora en la calidad de la interacción paciente-médico.
    \subsection{Fases}
    \subsubsection{Fase 1: Revisión de la literatura e investigación de campo.}
    En esta fase se realizará una revisión literaria para generar una perspectiva teórica sobre las tecnologías actuales con relación al historial médico de un paciente, los beneficios en la ciencia médica, las implicaciones legales y éticas y esquemas de seguridad para el manejo, auditoría y trazabilidad de la información. Se obtendrán antecedentes, casos de uso actuales, investigación del estado del arte de modelos similares y se generará un marco teórico que sustente la investigación.

    Además, se llevará a cabo una investigación de campo para identificar las necesidades y expectativas de los pacientes y médicos en relación al historial médico. Se realizarán entrevistas, encuestas y estudio de sombra a pacientes y médicos para obtener información sobre sus experiencias actuales con el historial médico y los procesos y procedimientos para la consulta. Así como sus expectativas y necesidades en cuanto a la gestión de esta información.
    \subsubsection{Fase 2: Diseño del modelo}
    Con la información recolectada, se realizará el diseño del  modelo de datos que se utilizará para la implementación del sistema de gestión de historial médico. Se definirán las entidades, atributos y relaciones necesarias para representar la información del paciente, así como los mecanismos de seguridad y trazabilidad requeridos. Se diseñará la arquitectura e interfases que permitan la interacción entre el paciente y el médico, así como la auditoría de las acciones realizadas sobre el historial médico.
    \subsubsection{Fase 3: Implementación del modelo}
    Implementación del modelo de datos diseñado en la fase anterior. Se desarrollarán las bases de datos necesarias, así como las aplicaciones y herramientas que permitan la gestión del historial médico. Se realizarán pruebas para verificar el correcto funcionamiento del sistema y su capacidad para mejorar la calidad de la interacción paciente-médico.
    \subsubsection{Fase 4: Evaluación del modelo}
    Evaluación del modelo implementado con la recolección de datos sobre su uso y efectividad. Se realizarán encuestas y entrevistas a pacientes y médicos para obtener retroalimentación sobre la experiencia de uso del sistema, así como métricas de calidad en la interacción paciente-médico. Se analizarán los resultados obtenidos para determinar si el modelo cumple con los objetivos planteados y si mejora la calidad del diagnóstico médico.
    \subsubsection{Fase 5: Conclusiones y recomendaciones}
    En esta fase se elaborarán las conclusiones de la investigación basadas en los resultados obtenidos en la fase de evaluación. Se identificarán las fortalezas y debilidades del modelo implementado, así como las oportunidades de mejora. Se realizarán recomendaciones para futuras investigaciones y para la implementación de mejoras en el sistema de gestión de historial médico.    
    \subsection{Participantes}
    En el proyecto se requiere de participaciones de actores multidisciplinarios, En un principio se requiere la colaboración de médicos y pacientes para la recolección de datos en la fase de investigación de campo. Se buscará la participación de médicos especialistas en diferentes áreas de la salud, así como pacientes con diferentes condiciones médicas, para obtener una perspectiva amplia sobre las necesidades y expectativas en relación al historial médico.

    Además, requerirá la asesoría legal y ética por parte de expertos en derecho médico, ética y privacidad de datos, para garantizar que el modelo propuesto cumpla con las normativas y regulaciones vigentes en materia de protección de datos personales y confidencialidad del historial médico.

    Asimismo, se buscará la colaboración de expertos en tecnología de la información y desarrollo de software para la implementación del modelo de datos y las aplicaciones necesarias para la gestión del historial médico. Se requerirá personal con experiencia en bases de datos, desarrollo de aplicaciones web y móviles, así como en seguridad informática.
    \subsection{Instrumentos}
    Para la recolección de datos en la fase de investigación de campo se utilizarán diferentes instrumentos, como encuestas estructuradas, entrevistas semiestructuradas y estudio de sombra. Las encuestas se diseñarán para obtener información cuantitativa sobre las experiencias y expectativas de los pacientes y médicos en relación al historial médico. Las entrevistas se utilizarán para profundizar en las percepciones y opiniones de los participantes sobre el modelo propuesto y su viabilidad. El estudio de sombra permitirá observar directamente los procesos y procedimientos actuales en la consulta médica y la gestión del historial médico.
    
    \subsection{Procedimiento y escenario}
    El procedimiento de la investigación se llevará a cabo de acuerdo a las fases descritas anteriormente. En primer lugar, se realizará la revisión de la literatura y la investigación de campo para obtener información sobre las necesidades y expectativas de los pacientes y médicos. A continuación, se diseñará el modelo de datos y se implementará en un entorno controlado para realizar pruebas y ajustes necesarios.

    Posteriormente, se evaluará el modelo implementado mediante la recolección de datos sobre su uso y efectividad, así como la retroalimentación de los participantes. Finalmente, se elaborarán las conclusiones y recomendaciones basadas en los resultados obtenidos.

    El escenario de la investigación se desarrollará en un entorno hospitalario o clínico, donde se contará con la colaboración de médicos y pacientes para la recolección de datos. Se garantizará el cumplimiento de las normativas éticas y legales en materia de protección de datos personales y confidencialidad del historial médico, así como la seguridad de la información manejada en el sistema propuesto.

    